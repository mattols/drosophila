% Options for packages loaded elsewhere
\PassOptionsToPackage{unicode}{hyperref}
\PassOptionsToPackage{hyphens}{url}
%
\documentclass[
]{article}
\usepackage{amsmath,amssymb}
\usepackage{lmodern}
\usepackage{iftex}
\ifPDFTeX
  \usepackage[T1]{fontenc}
  \usepackage[utf8]{inputenc}
  \usepackage{textcomp} % provide euro and other symbols
\else % if luatex or xetex
  \usepackage{unicode-math}
  \defaultfontfeatures{Scale=MatchLowercase}
  \defaultfontfeatures[\rmfamily]{Ligatures=TeX,Scale=1}
\fi
% Use upquote if available, for straight quotes in verbatim environments
\IfFileExists{upquote.sty}{\usepackage{upquote}}{}
\IfFileExists{microtype.sty}{% use microtype if available
  \usepackage[]{microtype}
  \UseMicrotypeSet[protrusion]{basicmath} % disable protrusion for tt fonts
}{}
\makeatletter
\@ifundefined{KOMAClassName}{% if non-KOMA class
  \IfFileExists{parskip.sty}{%
    \usepackage{parskip}
  }{% else
    \setlength{\parindent}{0pt}
    \setlength{\parskip}{6pt plus 2pt minus 1pt}}
}{% if KOMA class
  \KOMAoptions{parskip=half}}
\makeatother
\usepackage{xcolor}
\usepackage[margin=1in]{geometry}
\usepackage{color}
\usepackage{fancyvrb}
\newcommand{\VerbBar}{|}
\newcommand{\VERB}{\Verb[commandchars=\\\{\}]}
\DefineVerbatimEnvironment{Highlighting}{Verbatim}{commandchars=\\\{\}}
% Add ',fontsize=\small' for more characters per line
\usepackage{framed}
\definecolor{shadecolor}{RGB}{248,248,248}
\newenvironment{Shaded}{\begin{snugshade}}{\end{snugshade}}
\newcommand{\AlertTok}[1]{\textcolor[rgb]{0.94,0.16,0.16}{#1}}
\newcommand{\AnnotationTok}[1]{\textcolor[rgb]{0.56,0.35,0.01}{\textbf{\textit{#1}}}}
\newcommand{\AttributeTok}[1]{\textcolor[rgb]{0.77,0.63,0.00}{#1}}
\newcommand{\BaseNTok}[1]{\textcolor[rgb]{0.00,0.00,0.81}{#1}}
\newcommand{\BuiltInTok}[1]{#1}
\newcommand{\CharTok}[1]{\textcolor[rgb]{0.31,0.60,0.02}{#1}}
\newcommand{\CommentTok}[1]{\textcolor[rgb]{0.56,0.35,0.01}{\textit{#1}}}
\newcommand{\CommentVarTok}[1]{\textcolor[rgb]{0.56,0.35,0.01}{\textbf{\textit{#1}}}}
\newcommand{\ConstantTok}[1]{\textcolor[rgb]{0.00,0.00,0.00}{#1}}
\newcommand{\ControlFlowTok}[1]{\textcolor[rgb]{0.13,0.29,0.53}{\textbf{#1}}}
\newcommand{\DataTypeTok}[1]{\textcolor[rgb]{0.13,0.29,0.53}{#1}}
\newcommand{\DecValTok}[1]{\textcolor[rgb]{0.00,0.00,0.81}{#1}}
\newcommand{\DocumentationTok}[1]{\textcolor[rgb]{0.56,0.35,0.01}{\textbf{\textit{#1}}}}
\newcommand{\ErrorTok}[1]{\textcolor[rgb]{0.64,0.00,0.00}{\textbf{#1}}}
\newcommand{\ExtensionTok}[1]{#1}
\newcommand{\FloatTok}[1]{\textcolor[rgb]{0.00,0.00,0.81}{#1}}
\newcommand{\FunctionTok}[1]{\textcolor[rgb]{0.00,0.00,0.00}{#1}}
\newcommand{\ImportTok}[1]{#1}
\newcommand{\InformationTok}[1]{\textcolor[rgb]{0.56,0.35,0.01}{\textbf{\textit{#1}}}}
\newcommand{\KeywordTok}[1]{\textcolor[rgb]{0.13,0.29,0.53}{\textbf{#1}}}
\newcommand{\NormalTok}[1]{#1}
\newcommand{\OperatorTok}[1]{\textcolor[rgb]{0.81,0.36,0.00}{\textbf{#1}}}
\newcommand{\OtherTok}[1]{\textcolor[rgb]{0.56,0.35,0.01}{#1}}
\newcommand{\PreprocessorTok}[1]{\textcolor[rgb]{0.56,0.35,0.01}{\textit{#1}}}
\newcommand{\RegionMarkerTok}[1]{#1}
\newcommand{\SpecialCharTok}[1]{\textcolor[rgb]{0.00,0.00,0.00}{#1}}
\newcommand{\SpecialStringTok}[1]{\textcolor[rgb]{0.31,0.60,0.02}{#1}}
\newcommand{\StringTok}[1]{\textcolor[rgb]{0.31,0.60,0.02}{#1}}
\newcommand{\VariableTok}[1]{\textcolor[rgb]{0.00,0.00,0.00}{#1}}
\newcommand{\VerbatimStringTok}[1]{\textcolor[rgb]{0.31,0.60,0.02}{#1}}
\newcommand{\WarningTok}[1]{\textcolor[rgb]{0.56,0.35,0.01}{\textbf{\textit{#1}}}}
\usepackage{graphicx}
\makeatletter
\def\maxwidth{\ifdim\Gin@nat@width>\linewidth\linewidth\else\Gin@nat@width\fi}
\def\maxheight{\ifdim\Gin@nat@height>\textheight\textheight\else\Gin@nat@height\fi}
\makeatother
% Scale images if necessary, so that they will not overflow the page
% margins by default, and it is still possible to overwrite the defaults
% using explicit options in \includegraphics[width, height, ...]{}
\setkeys{Gin}{width=\maxwidth,height=\maxheight,keepaspectratio}
% Set default figure placement to htbp
\makeatletter
\def\fps@figure{htbp}
\makeatother
\setlength{\emergencystretch}{3em} % prevent overfull lines
\providecommand{\tightlist}{%
  \setlength{\itemsep}{0pt}\setlength{\parskip}{0pt}}
\setcounter{secnumdepth}{-\maxdimen} % remove section numbering
\ifLuaTeX
  \usepackage{selnolig}  % disable illegal ligatures
\fi
\IfFileExists{bookmark.sty}{\usepackage{bookmark}}{\usepackage{hyperref}}
\IfFileExists{xurl.sty}{\usepackage{xurl}}{} % add URL line breaks if available
\urlstyle{same} % disable monospaced font for URLs
\hypersetup{
  pdftitle={Drosophila-climatology},
  pdfauthor={Sam Curnow research project - Matt Olson},
  hidelinks,
  pdfcreator={LaTeX via pandoc}}

\title{Drosophila-climatology}
\author{Sam Curnow research project - Matt Olson}
\date{2023-09-29}

\begin{document}
\maketitle

\hypertarget{terra-and-gridded-data}{%
\subsubsection{Terra and gridded data}\label{terra-and-gridded-data}}

Most climate datasets are in a gridded format. The latest and best R
package to work with gridded data is the
\href{https://rspatial.github.io/terra/}{Terra package}. Libraries like
this require additional, external geospatial libraries installed on your
OS - e.g.~GDAL, GEOS, and PROJ.4. Installing Terra will vary based on
your OS. \textbf{Be sure to read} the documentation so that you properly
install any dependencies.

\emph{Working with geospatial libraries can be tricky, and I am willing
to help you troubleshoot if you run into issues.}

\begin{Shaded}
\begin{Highlighting}[]
\FunctionTok{library}\NormalTok{(terra);}\FunctionTok{library}\NormalTok{(dplyr)}
\end{Highlighting}
\end{Shaded}

I'll also read in the csv file you shared and filter for a few key
variables. Note you'll want to set your working directory or work from a
project.

\begin{Shaded}
\begin{Highlighting}[]
\CommentTok{\# read in data {-} omit NA values}
\NormalTok{df }\OtherTok{\textless{}{-}} \FunctionTok{read.csv}\NormalTok{(}\StringTok{"climate\_data\_drosophila\_June.csv"}\NormalTok{)}
\NormalTok{df2 }\OtherTok{\textless{}{-}}\NormalTok{ df }\SpecialCharTok{\%\textgreater{}\%}\NormalTok{ dplyr}\SpecialCharTok{::}\FunctionTok{select}\NormalTok{(Species, Subgenus, MbDNA\_Male, MbDNA\_Female, Tmax,Tmin) }\SpecialCharTok{\%\textgreater{}\%} \FunctionTok{na.omit}\NormalTok{()}
\FunctionTok{head}\NormalTok{(df2)}
\end{Highlighting}
\end{Shaded}

\begin{verbatim}
##                    Species   Subgenus MbDNA_Male MbDNA_Female  Tmax  Tmin
## 3  Drosophila_acanthoptera                 175.4        173.9 34.07  7.01
## 4       Drosophila_affinis Sophophora      165.6        168.7 35.06  5.49
## 6     Drosophila_algonquin Sophophora      170.0        156.2 34.81 13.14
## 8     Drosophila_ananassae Sophophora      179.9        169.0 30.13  6.72
## 9        Drosophila_anceps Drosophila      171.8        159.1 29.14  9.03
## 10    Drosophila_arawakana Drosophila      187.5        165.5 28.80  3.60
\end{verbatim}

\hypertarget{bioclimatic-data}{%
\subsubsection{Bioclimatic data}\label{bioclimatic-data}}

\href{https://www.worldclim.org/}{WorldClim} hosts several easy to
access datasets. Although these are not necessarily the most accurate
for some studies, they should be perfectly sufficient for others. R has
an easy-to-use library for accessing basic geographic and bioclimatic
global variables. Climate-related variables are generally averages over
a 30-year time-period (1970-2000).

\begin{Shaded}
\begin{Highlighting}[]
\FunctionTok{library}\NormalTok{(geodata)}
\CommentTok{\# define output folder for data download}
\ControlFlowTok{if}\NormalTok{ (}\SpecialCharTok{!}\FunctionTok{file.exists}\NormalTok{(}\StringTok{"data"}\NormalTok{))\{}\FunctionTok{dir.create}\NormalTok{(}\StringTok{"data"}\NormalTok{)\}}
\NormalTok{outpath }\OtherTok{=} \StringTok{"data"}
\CommentTok{\# download WorldClim global bioclimatic variables}
\NormalTok{bioclim19 }\OtherTok{\textless{}{-}} \FunctionTok{worldclim\_global}\NormalTok{(}\StringTok{\textquotesingle{}bio\textquotesingle{}}\NormalTok{, }\AttributeTok{res=}\DecValTok{10}\NormalTok{, }\AttributeTok{path=}\NormalTok{outpath)}
\FunctionTok{cat}\NormalTok{(}\StringTok{"Object has"}\NormalTok{,}\FunctionTok{nlyr}\NormalTok{(bioclim19),}\StringTok{"geographic layers"}\NormalTok{) }
\end{Highlighting}
\end{Shaded}

\begin{verbatim}
## Object has 19 geographic layers
\end{verbatim}

\begin{Shaded}
\begin{Highlighting}[]
\FunctionTok{plot}\NormalTok{(}\FunctionTok{subset}\NormalTok{(bioclim19, }\DecValTok{12}\NormalTok{), }\AttributeTok{main=}\StringTok{"BIO12 = Annual Precipitation"}\NormalTok{)}
\end{Highlighting}
\end{Shaded}

\includegraphics{Drosophila-climatology_files/figure-latex/unnamed-chunk-3-1.pdf}

We can further subset these variables into individual layers

\begin{Shaded}
\begin{Highlighting}[]
\CommentTok{\# subset variables of interest}
\NormalTok{tmax }\OtherTok{\textless{}{-}} \FunctionTok{subset}\NormalTok{(bioclim19, }\DecValTok{5}\NormalTok{);}\FunctionTok{names}\NormalTok{(tmax) }\OtherTok{\textless{}{-}} \StringTok{"BIO5 {-} Max Temperature of Warmest Month"}
\NormalTok{tmin }\OtherTok{\textless{}{-}} \FunctionTok{subset}\NormalTok{(bioclim19, }\DecValTok{6}\NormalTok{);}\FunctionTok{names}\NormalTok{(tmin) }\OtherTok{\textless{}{-}} \StringTok{"BIO6 {-} Min Temperature of Warmest Month"}
\FunctionTok{plot}\NormalTok{(}\FunctionTok{c}\NormalTok{(tmin,tmax), }\AttributeTok{col=}\FunctionTok{rev}\NormalTok{(}\FunctionTok{heat.colors}\NormalTok{(}\DecValTok{15}\NormalTok{)))}
\end{Highlighting}
\end{Shaded}

\includegraphics{Drosophila-climatology_files/figure-latex/unnamed-chunk-4-1.pdf}

\hypertarget{subsetting-geographic-datasets}{%
\subsubsection{Subsetting geographic
datasets}\label{subsetting-geographic-datasets}}

Fortunately your task is relatively simple. We can just subset these
gridded datasets based on the temperature thresholds in your csv file.

There are a few steps involved so I've created two functions. The code
could probably be more efficient, but this is what I wrote. Several
spatial functions are applied to the species in the csv file.

\begin{Shaded}
\begin{Highlighting}[]
\CommentTok{\# function to apply spatial bounds for all species}
\NormalTok{spat\_filter }\OtherTok{\textless{}{-}} \ControlFlowTok{function}\NormalTok{(csv.file, grid.var, var.name, }\AttributeTok{bounds=}\FunctionTok{c}\NormalTok{(}\StringTok{"lower"}\NormalTok{,}\StringTok{"upper"}\NormalTok{))\{}
\NormalTok{  grid.cpy }\OtherTok{\textless{}{-}} \FunctionTok{rast}\NormalTok{(}\FunctionTok{replicate}\NormalTok{(}\FunctionTok{nrow}\NormalTok{(csv.file),grid.var))}
  \FunctionTok{names}\NormalTok{(grid.cpy) }\OtherTok{\textless{}{-}}\NormalTok{ csv.file}\SpecialCharTok{$}\NormalTok{Species }\CommentTok{\# rename layers}
  \CommentTok{\# apply threshold from bounds arg}
\NormalTok{  lowval}\OtherTok{=}\SpecialCharTok{{-}}\ConstantTok{Inf}\NormalTok{;upval}\OtherTok{=}\SpecialCharTok{{-}}\ConstantTok{Inf}
  \ControlFlowTok{if}\NormalTok{ (bounds}\SpecialCharTok{==}\StringTok{\textquotesingle{}lower\textquotesingle{}}\NormalTok{)\{}
\NormalTok{    t.dist }\OtherTok{\textless{}{-}} \FunctionTok{rast}\NormalTok{(}\FunctionTok{sapply}\NormalTok{(}\DecValTok{1}\SpecialCharTok{:}\FunctionTok{nrow}\NormalTok{(csv.file), }\ControlFlowTok{function}\NormalTok{(x) }\FunctionTok{clamp}\NormalTok{(}\FunctionTok{subset}\NormalTok{(grid.cpy, x), }\AttributeTok{lower=}\NormalTok{csv.file[,var.name][x], }\AttributeTok{value=}\ConstantTok{FALSE}\NormalTok{)))\} }
  \ControlFlowTok{else}\NormalTok{\{}
\NormalTok{    t.dist }\OtherTok{\textless{}{-}} \FunctionTok{rast}\NormalTok{(}\FunctionTok{sapply}\NormalTok{(}\DecValTok{1}\SpecialCharTok{:}\FunctionTok{nrow}\NormalTok{(csv.file), }\ControlFlowTok{function}\NormalTok{(x) }\FunctionTok{clamp}\NormalTok{(}\FunctionTok{subset}\NormalTok{(grid.cpy, x), }\AttributeTok{upper=}\NormalTok{csv.file[,var.name][x], }\AttributeTok{value=}\ConstantTok{FALSE}\NormalTok{)))\} }
  \FunctionTok{return}\NormalTok{(t.dist)}
\NormalTok{\}}

\CommentTok{\# function to combine layers (if needed)}
\NormalTok{grid\_comb }\OtherTok{\textless{}{-}}  \ControlFlowTok{function}\NormalTok{(stack.a, stack.b, original.layer.a)\{}
  \CommentTok{\# boolean operator over all land surface where both conditions meet}
\NormalTok{  comb.grid }\OtherTok{\textless{}{-}} \SpecialCharTok{!}\FunctionTok{is.na}\NormalTok{(stack.a) }\SpecialCharTok{\&} \SpecialCharTok{!}\FunctionTok{is.na}\NormalTok{(stack.b) }\SpecialCharTok{\&} \SpecialCharTok{!}\FunctionTok{is.na}\NormalTok{(}\FunctionTok{rast}\NormalTok{(}\FunctionTok{replicate}\NormalTok{(}\FunctionTok{nlyr}\NormalTok{(stack.a),original.layer.a)))}
  \FunctionTok{return}\NormalTok{(comb.grid)}
\NormalTok{\}}
\end{Highlighting}
\end{Shaded}

\hypertarget{mapping-results}{%
\subsubsection{Mapping results}\label{mapping-results}}

Now we can run these functions and plot the results for one of the
variables. I'll just stick to simple map plots.

\begin{Shaded}
\begin{Highlighting}[]
\CommentTok{\# run the function for both variables}
\NormalTok{tmax.stack }\OtherTok{\textless{}{-}}  \FunctionTok{spat\_filter}\NormalTok{(df2, tmax, }\StringTok{"Tmax"}\NormalTok{, }\AttributeTok{bounds=}\StringTok{"upper"}\NormalTok{)}
\NormalTok{tmin.stack }\OtherTok{\textless{}{-}}  \FunctionTok{spat\_filter}\NormalTok{(df2, tmin, }\StringTok{"Tmin"}\NormalTok{, }\AttributeTok{bounds=}\StringTok{"lower"}\NormalTok{)}

\CommentTok{\# tmin distribution of first four species}
\FunctionTok{plot}\NormalTok{(tmin.stack[[}\DecValTok{1}\SpecialCharTok{:}\DecValTok{4}\NormalTok{]], }\AttributeTok{col=}\FunctionTok{rev}\NormalTok{(}\FunctionTok{heat.colors}\NormalTok{(}\DecValTok{15}\NormalTok{)))}
\end{Highlighting}
\end{Shaded}

\includegraphics{Drosophila-climatology_files/figure-latex/unnamed-chunk-6-1.pdf}

And combine the layers to show where all conditions are met.

\begin{Shaded}
\begin{Highlighting}[]
\CommentTok{\# combine layers and plot distribution}
\NormalTok{t.dist }\OtherTok{\textless{}{-}} \FunctionTok{grid\_comb}\NormalTok{(tmax.stack,tmin.stack,tmax)}
\CommentTok{\# plot the distribution for the first four}
\FunctionTok{plot}\NormalTok{(t.dist[[}\DecValTok{1}\SpecialCharTok{:}\DecValTok{4}\NormalTok{]])}
\end{Highlighting}
\end{Shaded}

\includegraphics{Drosophila-climatology_files/figure-latex/unnamed-chunk-7-1.pdf}

You can also add country boundaries to give greater context to a map.
This function is within the same package we loaded above.

\begin{Shaded}
\begin{Highlighting}[]
\CommentTok{\# plot first species with world map}
\FunctionTok{library}\NormalTok{(ggplot2)}
\NormalTok{worldmap }\OtherTok{\textless{}{-}} \FunctionTok{world}\NormalTok{(}\AttributeTok{resolution=}\DecValTok{5}\NormalTok{,}\AttributeTok{level=}\DecValTok{0}\NormalTok{,outpath,}\AttributeTok{version=}\StringTok{"latest"}\NormalTok{)}
\FunctionTok{plot}\NormalTok{(}\FunctionTok{subset}\NormalTok{(tmin.stack,}\DecValTok{1}\NormalTok{), }\AttributeTok{col=}\FunctionTok{rev}\NormalTok{(}\FunctionTok{heat.colors}\NormalTok{(}\DecValTok{15}\NormalTok{)), }\AttributeTok{main=}\FunctionTok{paste}\NormalTok{(}\StringTok{"Tmin threshold for"}\NormalTok{,df2}\SpecialCharTok{$}\NormalTok{Species[}\DecValTok{1}\NormalTok{]) )}
\FunctionTok{plot}\NormalTok{(worldmap, }\AttributeTok{add=}\ConstantTok{TRUE}\NormalTok{, }\AttributeTok{lwd=}\FloatTok{0.5}\NormalTok{, }\AttributeTok{border=}\FunctionTok{alpha}\NormalTok{(}\FunctionTok{rgb}\NormalTok{(}\DecValTok{0}\NormalTok{,}\DecValTok{0}\NormalTok{,}\DecValTok{0}\NormalTok{), }\FloatTok{0.5}\NormalTok{))}
\end{Highlighting}
\end{Shaded}

\includegraphics{Drosophila-climatology_files/figure-latex/unnamed-chunk-8-1.pdf}

\hypertarget{plotting-results}{%
\subsubsection{Plotting results}\label{plotting-results}}

Finally, you can generate some statistics about the spatial distribution
of these species.

\begin{Shaded}
\begin{Highlighting}[]
\CommentTok{\# generate some statistics from the first species only}
\FunctionTok{crds}\NormalTok{(t.dist[[}\DecValTok{1}\NormalTok{]], }\AttributeTok{df=}\ConstantTok{TRUE}\NormalTok{) }\SpecialCharTok{\%\textgreater{}\%} \FunctionTok{filter}\NormalTok{(}\FunctionTok{values}\NormalTok{(t.dist[[}\DecValTok{1}\NormalTok{]])}\SpecialCharTok{==}\DecValTok{1}\NormalTok{) }\SpecialCharTok{\%\textgreater{}\%} 
  \FunctionTok{summarize}\NormalTok{(}\AttributeTok{n=}\FunctionTok{n}\NormalTok{(), }\AttributeTok{latmean=}\FunctionTok{mean}\NormalTok{(y), }\AttributeTok{latmax=}\FunctionTok{max}\NormalTok{(y), }\AttributeTok{latmin=}\FunctionTok{min}\NormalTok{(y) )}
\end{Highlighting}
\end{Shaded}

\begin{verbatim}
##       n   latmean   latmax    latmin
## 1 96099 -2.541921 48.41667 -48.08333
\end{verbatim}

We can build a function and calculate these same stats over the entire
dataset, re-inserting the values into our dataframe. Again, the code
could probably be more efficient. This step may take a minute, so you
may want to refill your coffee.

\begin{Shaded}
\begin{Highlighting}[]
\CommentTok{\# function to extract stats from a given layer}
\NormalTok{lyr\_summary }\OtherTok{\textless{}{-}} \ControlFlowTok{function}\NormalTok{(lyr)\{}
\NormalTok{  rowinfo }\OtherTok{\textless{}{-}} \FunctionTok{crds}\NormalTok{(lyr, }\AttributeTok{df=}\ConstantTok{TRUE}\NormalTok{) }\SpecialCharTok{\%\textgreater{}\%} \FunctionTok{filter}\NormalTok{(}\FunctionTok{values}\NormalTok{(lyr)}\SpecialCharTok{==}\DecValTok{1}\NormalTok{) }\SpecialCharTok{\%\textgreater{}\%} \FunctionTok{summarize}\NormalTok{(}\AttributeTok{n=}\FunctionTok{n}\NormalTok{(), }\AttributeTok{latmean=}\FunctionTok{mean}\NormalTok{(y), }\AttributeTok{latmax=}\FunctionTok{max}\NormalTok{(y), }\AttributeTok{latmin=}\FunctionTok{min}\NormalTok{(y) )}
  \FunctionTok{return}\NormalTok{(rowinfo)}
\NormalTok{\}}
\CommentTok{\# apply this to all layers}
\NormalTok{t.df2 }\OtherTok{\textless{}{-}} \FunctionTok{sapply}\NormalTok{(}\DecValTok{1}\SpecialCharTok{:}\FunctionTok{nlyr}\NormalTok{(t.dist), }\ControlFlowTok{function}\NormalTok{(x) }\FunctionTok{lyr\_summary}\NormalTok{(t.dist[[x]])  )}
\CommentTok{\# insert layers into dataframe and a clean}
\NormalTok{df3 }\OtherTok{\textless{}{-}} \FunctionTok{as.data.frame}\NormalTok{(}\FunctionTok{t}\NormalTok{(t.df2)) }\SpecialCharTok{\%\textgreater{}\%} \FunctionTok{mutate}\NormalTok{(}\AttributeTok{Species =}\NormalTok{ df2}\SpecialCharTok{$}\NormalTok{Species, }\AttributeTok{.before=}\NormalTok{n) }\SpecialCharTok{\%\textgreater{}\%} \FunctionTok{full\_join}\NormalTok{(df2, }\AttributeTok{by=}\StringTok{\textquotesingle{}Species\textquotesingle{}}\NormalTok{) }\SpecialCharTok{\%\textgreater{}\%}
  \FunctionTok{mutate}\NormalTok{(}\AttributeTok{n =} \FunctionTok{unlist}\NormalTok{(n), }\AttributeTok{latmean =} \FunctionTok{unlist}\NormalTok{(latmean),}\AttributeTok{latmax =} \FunctionTok{unlist}\NormalTok{(latmax),}\AttributeTok{latmin =} \FunctionTok{unlist}\NormalTok{(latmin) ) }
\FunctionTok{head}\NormalTok{(df3)}
\end{Highlighting}
\end{Shaded}

\begin{verbatim}
##                   Species      n   latmean   latmax    latmin   Subgenus
## 1 Drosophila_acanthoptera  96099 -2.541921 48.41667 -48.08333           
## 2      Drosophila_affinis 117630 -2.697064 57.58333 -48.08333 Sophophora
## 3    Drosophila_algonquin  74469 -1.119279 32.58333 -31.58333 Sophophora
## 4    Drosophila_ananassae  24717 -5.506854 49.91667 -48.08333 Sophophora
## 5       Drosophila_anceps  11212 -4.743222 39.91667 -40.41667 Drosophila
## 6    Drosophila_arawakana  18680 -5.992666 63.08333 -50.91667 Drosophila
##   MbDNA_Male MbDNA_Female  Tmax  Tmin
## 1      175.4        173.9 34.07  7.01
## 2      165.6        168.7 35.06  5.49
## 3      170.0        156.2 34.81 13.14
## 4      179.9        169.0 30.13  6.72
## 5      171.8        159.1 29.14  9.03
## 6      187.5        165.5 28.80  3.60
\end{verbatim}

And a quick plot:

\begin{Shaded}
\begin{Highlighting}[]
\CommentTok{\#quick plot of two variables}
\NormalTok{df3 }\SpecialCharTok{\%\textgreater{}\%}
  \FunctionTok{ggplot}\NormalTok{(}\FunctionTok{aes}\NormalTok{(}\AttributeTok{x=}\NormalTok{latmax, }\AttributeTok{y=}\NormalTok{MbDNA\_Male, }\AttributeTok{colour=}\NormalTok{Tmin)) }\SpecialCharTok{+} 
  \FunctionTok{geom\_point}\NormalTok{(}\FunctionTok{aes}\NormalTok{(}\AttributeTok{size=}\NormalTok{n)) }\SpecialCharTok{+} \FunctionTok{scale\_color\_continuous}\NormalTok{(}\AttributeTok{type =} \StringTok{\textquotesingle{}viridis\textquotesingle{}}\NormalTok{) }\SpecialCharTok{+} 
  \FunctionTok{geom\_smooth}\NormalTok{(}\AttributeTok{method=}\StringTok{\textquotesingle{}lm\textquotesingle{}}\NormalTok{, }\AttributeTok{formula=}\StringTok{\textquotesingle{}y\textasciitilde{}x\textquotesingle{}}\NormalTok{) }\SpecialCharTok{+} \FunctionTok{theme\_bw}\NormalTok{()}
\end{Highlighting}
\end{Shaded}

\includegraphics{Drosophila-climatology_files/figure-latex/unnamed-chunk-11-1.pdf}

And all variables that we selected:

\begin{Shaded}
\begin{Highlighting}[]
\NormalTok{df3 }\SpecialCharTok{\%\textgreater{}\%}\NormalTok{ dplyr}\SpecialCharTok{::}\FunctionTok{select}\NormalTok{(latmean, latmax, latmin, MbDNA\_Male, MbDNA\_Female, Tmax, Tmin) }\SpecialCharTok{\%\textgreater{}\%} \FunctionTok{pairs}\NormalTok{()}
\end{Highlighting}
\end{Shaded}

\includegraphics{Drosophila-climatology_files/figure-latex/unnamed-chunk-12-1.pdf}

I'll let you take it from here Sam! Hopefully this is what you had in
mind when you asked for help. You can replace other variables in the
steps above to include annual precipitation, and even monthly
temperature information. There are many other spatial questions that you
could ask with such a dataset. \textbf{Good luck!}

\end{document}
